\documentclass[a4paper,14pt]{article}
\input{AESh.sty}
\usepackage{caption}
\usepackage{pdfpages}


\begin{document}
\begin{titlepage}
\thispagestyle{empty}
\setcounter{page}{0}
\begin{center}{
\LargeМосковский государственый университет 

им. М.",В.",Ломоносова

Механико-математический факультет

Кафедра газовой и волновой динамики}
\end{center}
\vspace{120pt}
\begin{center}
\Large Ракитин Виталий Павлович.
\end{center}
\vspace{20pt}
\begin{center}{\LARGE Интегрирование системы ОДУ методом Рунге-Кутта.}
\end{center}

\vspace{80pt}
%\begin{flushleft}
%{\largeНаучный руководитель:

%\vspace{6pt}
%чл.-корр. РАН, проф. Трещев Д.",В.}
%\end{flushleft}
\vfill
\begin{center}
Москва, 2015 год.
\end{center}
\end{titlepage}

\tableofcontents
\newpage

\section{Постановка задачи}

{\bf Задача 2.6: } Решить задачу Коши и построить график траектории движения.
\[
\begin{cases}
	\begin{cases}
	\frac{d^2 x}{dt^2} = y(2 - x^2 - y^2); \\
	\frac{d^2 x}{dt^2} = -x(2 - x^2 - y^2);
    \end{cases}\\
	0 < t < 30; \\
	t = 0: x = 0, y =\alpha, \frac{d x}{dt} =\frac{d y}{dt} = 0;\\
	\alpha \in \{1.0, 0.1\}.
\end{cases}
\]

Данную систему из 2х уравнений второго порядка можно свести к эвкивалентной из 4х уравнений первого порядка. Сделаем замену $\frac{dx}{dt} = u$, $\frac{dy}{dt} = v$, получим
\[
\begin{cases}
	\begin{cases}
	\frac{d x}{d t} = u;\\
	\frac{d y}{d t} = v;\\
	\frac{d u}{d t} = y(2 - x^2 - y^2); \\
	\frac{d v}{d t} = -x(2 - x^2 - y^2);
    \end{cases}\\
	0 < t < 30; \\
	t = 0: x = 0, y =\alpha, u = v = 0;\\
	\alpha \in \{1.0, 0.1\}.
\end{cases}
\]

Так~как данная задача не~имеет аналитического решения, напишем компьютерную программу на~языке программирования~$C$ для~численного решения.

\section{Метод решения}
Для решения данной задачи применим метод Рунге - Кутты 5-го порядка с коэффициентами Дормана-Принса 5(4). Для~этого зафиксируем следующие числа 
\begin{center}
\begin{tabular}{|c|ccccccc|}
\hline
{\bf $\alpha$} & \multicolumn{7}{|c|}{{\bf $\beta$}}\\
\hline
    & & & & & & &\\
$0$ & & & & & & & \\ 
    & & & & & & &\\
$\frac{1}{5}$ &$\frac{1}{5}$ & & & & &  &\\
& & & & & &  &\\
$\frac{3}{10}$ &$\frac{3}{40}$ &$\frac{9}{40}$  & & & &  &\\ 
& & & & & &  &\\
$\frac{4}{5}$ &$\frac{44}{45}$ &$-\frac{56}{15}$  &  $\frac{32}{9}$& & & & \\
& & & & & &  &\\
$\frac{8}{9}$ &$\frac{19372}{6561}$ &$-\frac{25360}{2187}$  &  $\frac{64448}{6561}$& $-\frac{212}{729}$ & & & \\
& & & & & & &\\
$1$ &$\frac{9071}{3168}$ &$-\frac{355}{33}$  &  $\frac{46732}{5247}$& $\frac{49}{176}$ &  $-\frac{5103}{18656}$&  &\\
& & & & & & & \\
$1$ &$\frac{35}{384}$ &$0$  &  $\frac{500}{1113}$& $\frac{125}{192}$ &  $-\frac{2187}{6784}$&$\frac{11}{84}$ & \\
& & & & & & & \\
\hline
& & & & & & & \\
$p$ &$\frac{35}{384}$ &$0$  &  $\frac{500}{1113}$& $\frac{125}{192}$ &  $-\frac{2187}{6784}$&$\frac{11}{84}$ & 0 \\
& & & & & & & \\
\hline
& & & & & & & \\
$\bar p$ &$\frac{5179}{57600}$ &$0$  &  $\frac{7571}{16695}$& $\frac{393}{640}$ &  $-\frac{92097}{339200}$&$\frac{187}{2100}$&$\frac{1}{40}$  \\
& & & & & & & \\
\hline
\end{tabular}
\end{center}

Последовательно получим систему, где $y_n \in \{x_n, y_n, u_n, v_n\}$,
\[
	\begin{cases}
k_1(h) = h f \left(t_n, y_n\right);\\
k_2(h) = h f \left(t_n + \alpha_2 h,y_n+\beta_{21}k_1(h)\right);\\
\dots\\ 
k_q(h) = h f \left(t_n + \alpha_q h,y_n+ \sum\limits_{i=1}^{q-1} \beta_{qi}k_i(h)\right).\\
	\end{cases}\\
\]
которая после подставления чисел $\alpha_i,  p_i, \beta_i$ принимает вид
\[
\begin{cases}
k_1(h) = h f \left(t_n,y_n \right);\\
\\
k_2(h) = h f \left(t_n + \frac{1}{5}h,y_n+\frac{1}{4} k_1\right);\\
\\
k_3(h) = h f \left(t_n + \frac{3}{10}h,y_n+\frac{3}{40}k_1 + \frac{9}{40}k_2\right);\\
\\
k_4(h) = h f \left(t_n + \frac{4}{5}h, y_n + \frac{44}{45}k_1 - \frac{56}{15} k_2 + \frac{32}{9} k_3\right);\\
\\
k_5(h) = h f \left(t_n + \frac{8}{9}h,y_n+\frac{19372}{6561}k_1 - \frac{25360}{2187} k_2 +\frac{64448}{6561} k_3 - \frac{212}{729} k_4\right);\\
\\
k_6(h) = h f \left(t_n + h,y_n + \frac{9017}{3168}k_1 - \frac{355}{33}k_2  + \frac{46732}{5247} k_3 + \frac{49}{176} k_4 - \frac{5103}{18656} k_5 \right);\\
\\
k_7(h) = h f \left(t_n + h,y_n + \frac{35}{384}k_1 + \frac{500}{1113} k_3 + \frac{125}{192} k_4 - \frac{2187}{6784} k_5 + \frac{11}{84} k_6 \right).\\

\end{cases}\\
\]

\[
y_{n}(t+h) = y_n (t) + \sum_{i=1}^q {p_ik_i} = y_n (t) + \frac{35}{384}k_1 + \frac{500}{1113} k_3 + \frac{125}{192} k_4 - \frac{2187}{6784} k_5 + \frac{11}{84} k_6;
\]
\[
\bar {y}_{n}(t+h) = y_n (t) + \sum_{i=1}^q {\bar p_ik_i} = y_n (t) + \frac{517}{57600}k_1 + \frac{7571}{16695}k_3 +\frac{393}{640}k_4 - \frac{92097}{339200}k_5 + \frac{187}{2100}k_6 + \frac{1}{40}k_7. 
\]

Таким образом величина погрешности имеет порядок точности $O(h^5)$ и её можно оценить двумя способами. Во-первых, как~расстояние между двумя точками $A$ и $\bar A$
\[
Err = \sqrt{(x_n - \bar x_n)^2 + (y_n - \bar y_n)^2 + (u_n - \bar u_n)^2 + (v_n - \bar v_n)^2}. 
\]
Во-вторых, как максимум расстояний между соответствующими координатами, а именно
\[
Err = \max \left( x_n - \bar x_n,y_n - \bar y_n, u_n - \bar u_n, v_n - \bar v_n \right). 
\]
Для удобства будем использовать второй вариант.
\subsection{Горизонтальная процедура автоматического выбора шага}
Для того, чтобы избешать сильного отклонения от допустимого для корректных вычислений значения шага, напишем так же функцию, которая будет автоматически изменять наш шаг в течении всего времени расчётов.

Пусть $\e$ "--- это заранее заданная точность нашей погрешности $Err$, 
а значения корректирующих констант заданы следующим образом (их можно менять по усмотрению программиста)  
\[
FAC = 0.9, \qquad FACMIN = 0.5,\qquad FACMAX = 2.0.
\]
В таком случае формула для изменения шага будет представлена в следующем виде
\[
h_{new} = h \cdot \min \left(FACMAX, \max \left(FACMIN, \left(\frac{\e}{Err} \right)^{\frac{1}{1 + p}} \right)\right),
\]
где $p = 5$ "--- порядок алгоритма.
Таким образом, при расчётах будем производить коррекцию шага до тех пор, пока не добьёмся нужного порядка погрешности наших вычислений.
\section{Оценка корректности результата}
Прежде всего нам необходимо знать, можем ли мы доверять полученному результату. Для оценки корректности работы программы будем использовать нам необходимо понять, какова погрешность наших измерений. Для этого будем высчитывать два значения:
\begin{enumerate}
\item велечина отклонения на каждом шаге;
\item глобальная велечина отклонения.
\end{enumerate}
\subsection{Оценка локального отклонения (на шаге)}
Оценивать велечину отклонения на каждом шаге не имеет никакого смысла из-за значительного увелечения времени работы программы. Поэтому, для этой оценки проведём все вышеизложенные вычисления для трёх значений точности велечины отклонения, а именно $-7, -9$ и $-11$. Далее зафиксируем 4 момента времени: $\frac{T}{4}, \frac{T}{2}, \frac{3T}{4}, T$, где $T$ "--- полное время работы нашей программы. После этого вычислим все значение переменных в каждой из этих точек, и проверим следующее соотношение
\[
variation = \frac{y_{-11} - y_{-9}}{y_{-9} - y_{-7}} \approx 100^{\frac{s}{s+1}} = 100^{\frac{5}{5+1}} \approx  46.42 
\]

После проведения всех расчётов для $\alpha = 1.0$ были получены следующие величины
\[
\text{При $ t = 0.875 $ для переменной $x$: } \quad variation = 2.115109
\]
\[
\text{При  $t = 0.875 $ для переменной $y$: }\quad variation = 2.337501
\]
\[
\text{При $ t = 0.875 $ для переменной $u$: } \quad variation = 1.943925
\]
\[
\text{При  $t = 0.875 $ для переменной $v$: }\quad variation = 2.170464
\]
\[
\text{При  $t = 1.75$ для переменной $x$: } \quad variation = 1.880996
\]
\[
\text{При  $t = 1.75 $ для переменной $y$: }\quad variation = 1.996141
\]
\[
\text{При  $t = 1.75$ для переменной $u$: } \quad variation = 1.646851
\]
\[
\text{При  $t = 1.75 $ для переменной $v$: }\quad variation = 1.792992
\]
\[
\text{При  $t = 2.625 $ для переменной $x$: }\quad variation = 1.194104
\]
\[
\text{При  $t = 2.625 $ для переменной $y$: }\quad variation = 1.057303
\]
\[
\text{При  $t = 2.625 $ для переменной $u$: }\quad variation = 1.490886
\]
\[
\text{При  $t = 2.625 $ для переменной $v$: }\quad variation = 1.573281
\]
\[
\text{При  $t = 3.5 $ для переменной $x$: }\quad variation =  83.63506
\]
\[
\text{При  $t = 3.5 $ для переменной $y$: } \quad variation = 1.837813
\]
\[
\text{При  $t = 3.5 $ для переменной $u$: }\quad variation =  6.108553
\]
\[
\text{При  $t = 3.5 $ для переменной $v$: } \quad variation = 1.147372
\]

\subsection{Оценка глобальной погрешности.}
Для вычисления глобальной погрешности введём множество переменных $\delta_i:$ 
\[
 \delta_0 = 0, \qquad \delta_{k+1} = Err_{k} + \delta_{k} \cdot e^{\int\limits_{t_k}^{t_{k+1}} \mu(s) ds} 
\]
Интеграл в предыдущем выражении можно приблизить следующим образом
\[
\int\limits_{t_k}^{t_{k+1}} \mu(s) ds = (t_{k+1} - t_k) \cdot Hmax \left(\frac{J + J^T}{2} \right),
\]
где $J$ "--- матрица Якоби исходной системы дифференциальных уравнений, а $Hmax$ "--- функция, возвращающая максимальное собственное значение полученной матрицы. 

\[
J =
\left(
\begin{array}{cccc}
 0 & 0 & 1 & 0 \\
 0 & 0 & 0 & 1 \\
 -2 x y & -x^2-3 y^2+2 & 0 & 0 \\
 3 x^2+y^2-2 & 2 x y & 0 & 0 \\
\end{array}
\right)
\]
Соответственно, 
\[
A = \frac{J + J^T}{2} = 
\left(
\begin{array}{cccc}
 0 & 0 & \frac{1}{2} (1-2 x y) & \frac{1}{2} \left(3 x^2+y^2-2\right) \\
 0 & 0 & \frac{1}{2} \left(-x^2-3 y^2+2\right) & \frac{1}{2} (2 x y+1) \\
 \frac{1}{2} (1-2 x y) & \frac{1}{2} \left(-x^2-3 y^2+2\right) & 0 & 0 \\
 \frac{1}{2} \left(3 x^2+y^2-2\right) & \frac{1}{2} (2 x y+1) & 0 & 0 \\
\end{array}
\right)
\]
Тогда собственные значения матрицы $A$ будут выражаться в следующем виде
\[
\lambda_1 = -x^2-y^2-\sqrt{4 x^4+8 x^2 y^2-8 x^2+4 y^4-8 y^2+5}, \qquad \lambda_2 = x^2+y^2-\sqrt{4 x^4+8 x^2 y^2-8 x^2+4 y^4-8 y^2+5}
\]
\[
\lambda_3 =-x^2-y^2+\sqrt{4 x^4+8 x^2 y^2-8 x^2+4 y^4-8 y^2+5}, \qquad \lambda_4 = x^2+y^2 +\sqrt{4 x^4+8 x^2 y^2-8 x^2+4 y^4-8 y^2+5}.
\]
Несложно заметить, что максимальным собственным значением является именно $\lambda_4$, которое и будем использовать в дальнейших расчётах. Так же вспомним, что $ t_{k+1} - t_k = h$. Значит окончательные формулы для расчёта погрешности принимают вид
\[
\delta_0 = 0, \qquad\delta_{k+1} = Err_{k} + \delta_{k} \cdot e^{\lambda}, \qquad \lambda = \int\limits_{t_k}^{t_{k+1}} \mu(s) ds = h \cdot \left(x^2+y^2 +\sqrt{4 x^4+8 x^2 y^2-8 x^2+4 y^4-8 y^2+5} \right).
\]

\section{Результаты}
По вышеописанному алгоритму была написана компьютерная программа на языке программирования $C$. По результатам её работы было получено множество точек, составляющих траекторию движения системы, описанной данными уравениями.
Однако, опытным путём было выяснено, что для $\alpha = 1.0$. при стремлении $t \to 3.6524$ значения скоростей стремяться к бесконечности. Поэтому расчёты производились до предельного значения точности шага, который был выбран следующим образом $H_{DEFAULT} = 10^{-15}$ и расчёты происходили до соответствующего момента времени $t = 3.5$. После этого по описанному выше алгоритму были посчитаны величины погрешности, которые удовляют всем поставленым ранее ограничениям и указывают на то, что~полученная программа работает корректно.

Так же для проверки данная задача была решена с помощью пакета {\it Wolfram Mathematica 9} (см.~прилож.~\ref{Task1_Wolfram}). Отчётливо видно, что при запуске {\it Wolfram} выдаёт следющую ошибку:

 {\it "At t = 3.652401607212685, step size is effectively zero; singularity or stiff system suspected".}\\
Таким образом, аналагично нашей программе, был произведён анализ до момента времени $t = 3.5$.

Аналогично при $\alpha = 0.1$ критическим значением оказалось $t = 5$.
\subsection{Визуализация полученных результатов}
Для более наглядного представления полученных результатов была написана компьютерная программа на языке программирования $Python$, с помощью библиотеки $Matplotlib$, которая строит графики зависимости наших координат от времени, а так же друг от друга, а именно x(t), y(t) и y(x) (см.~рис.~\ref{py1}). 
Соответствующие графики для $\alpha = 1.0$ представлены на рис.~(\ref{myx-y}),~(\ref{myt-y}),~(\ref{myt-x}) и $\alpha = 0.1$ на ~(\ref{1myx-y}),~(\ref{1myt-y}),~(\ref{1myt-x})

\subsection{Анализ полученных результатов.}
Проведём анализ полученных результатов можно провести в три стадии.
\begin{enumerate}
\item Несложно заметить, что полученное графики полностью совпадают с теми, что полученны с помощью пакета {\it Wolfram Mathematica 9} (см.~прилож.~\ref{Task1_Wolfram} и рис.~(\ref{myx-y}),~(\ref{myt-y})~и~(\ref{myt-x})).
\item Из-за того, что максимальное собственное значение матрицы
\[
A = \frac{J + J^T}{2}
\]
быстро достигает больших значений, то экспонента, а соответственно и велечина глобальной погрешности, растут с крайне высокой скоростью, поэтому через какое-то время они переходят через точную верхнюю гран множества допустим значений соответствующего типа. Это явление делает невозможным подсчёт глобального отклонения. 
\item Из-за того, что значение скорости быстро убегает в бесконечность, количество шагов метода Рунге-Кутта так же слишком мало для~корректного подсчёта локального отклонения.

\begin{figure}[H]
\noindent\centering{
\includegraphics[width=110mm]{pictures/py1.png} 
  \caption{Код на языке $Python$ для построения графика $x(t)$.}
  \label{py1}}
\end{figure}
\begin{figure}[H]
\centering
    \includegraphics[width=100mm]{pictures/10x-y.pdf}
    \caption{Решение задачи 2.6 при $\alpha = 1.0$. График зависимости $y(x)$}
    \label{myx-y}
\end{figure}
\end{enumerate}
\begin{figure}[H]
\centering
    \includegraphics[width=100mm]{pictures/10t-y.pdf}
    \caption{Решение задачи 2.6 при $\alpha = 1.0$. График зависимости $y(t)$}
    \label{myt-y} 
\end{figure}
\begin{figure}[H]
\centering
    \includegraphics[width=110mm]{pictures/10t-x.pdf}
    \caption{Решение задачи 2.6 при $\alpha = 1.0$. График зависимости $x(t)$}
    \label{myt-x}
\end{figure}
\begin{figure}[H]
\centering
    \includegraphics[width=110mm]{pictures/01x-y.pdf}
    \caption{Решение задачи 2.6 при $\alpha = 0.1$. График зависимости $y(x)$}
    \label{1myx-y}
\end{figure}
\begin{figure}[H]
\centering
    \includegraphics[width=110mm]{pictures/01t-y.pdf}
    \caption{Решение задачи 2.6 при $\alpha = 0.1$. График зависимости $y(t)$}
    \label{1myt-y} 
\end{figure}
\begin{figure}[H]
\centering
    \includegraphics[width=110mm]{pictures/01t-x.pdf}
    \caption{Решение задачи 2.6 при $\alpha = 0.1$. График зависимости $x(t)$}
    \label{1myt-x}
\end{figure}

\section{Гармонический осциллятор}
Дабы отбросить все сомнения в полученных результатах, проверим работоспособность программы на более простом случае с заранее известным решением, а именно на гармоническом осцилляторе
\[
\begin{cases}
	\frac{d x}{d t} = y;\\
	\frac{d y}{d t} = -x;\\
	0 < t < 30; \\
	t = 0: x = 0, y = 8.
\end{cases}
\]

Визуализация численного решения данной задачи с помощью нашей программы представлена в виде графиков на рис.(\ref{t-x}), (\ref{t-y}) и (\ref{x-y}). Для удобства проверки дополнительно решим нашу задачу с помощью пакета {\it Wolfram Mathematica 9} (см.~рис.~\ref{Wolfram}).
Сравнивая полученные результаты можно заметить, что полученные решения абсолютно идентичны, на основе чего можно сделать вывод о корректности работы программы. Однако, для большей достоверности проверим так же численные оценки наших погрешностей.\\
\begin{enumerate}
\item Для вычисления {\bf глобальной погрешности} найдём матрицу Якоби нашей системы
\[
J = \left(
\begin{array}{cc}
 0 & 1 \\
 -1 & 0 \\
\end{array}
\right),
\qquad
A = \frac{J + J^T}{2} =
\left(
\begin{array}{cc}
 0 & 0 \\
 0 & 0 \\
\end{array}
\right)
\]
\[
\lambda_{1,2} = 0, \qquad \delta_0 = 0, \qquad \delta_{k+1} = Err_{k} + \delta_{k}.
\]
Таким образом были получены следюущие значения глобальной погрешности:
\[
\text{для точности погрешности -7-го порядка } \delta_k =  2.433796 \cdot 10^{-6}
\]
\[
\text{для точности погрешности -9-го порядка } \delta_k =  3.964618 \cdot 10^{-7}
\]
\[
\text{для точности погрешности -11-го порядка } \delta_k =  8.135620 \cdot 10^{-9}
\]

\item А оценка {\bf локального отклонения на шаге} для каждой точки $t = \{50, 100, 150, 200\}$ получилась равна в районе $100 \pm 2$.
\end{enumerate}

На основе выше сказанного можно сделать вывод, что полученная программа работает корректно.
\newpage
\begin{figure}[H]
\centering
    \includegraphics[width=100mm]{pictures/x-y.pdf}
    \caption{Решение задачи о математическом осцилляторе. График зависимости $y(x)$}
    \label{x-y}
\end{figure}
\begin{figure}[H]
\centering
    \includegraphics[width=100mm]{pictures/t-y.pdf}
    \caption{Решение задачи о математическом осцилляторе. График зависимости $y(t)$}
    \label{t-y} 
\end{figure}
\begin{figure}[H]
\centering
    \includegraphics[width=100mm]{pictures/t-x.pdf}
    \caption{Решение задачи о математическом осцилляторе. График зависимости $x(t)$}
    \label{t-x}
\end{figure}

%\begin{figure}[bh]
%\noindent\centering{
%\includegraphics[width=180mm]{oscillator.pdf} 
%  \caption{Решение задачи о математическом осцилляторе в Wolfram Mathematica 9.}
%  \label{Wolfram}}
%\end{figure}

\newpage
\section{Приложения}
\subsection{Решение задачи 2.6 Wolfram Mathematica 9.}
\label{Task1_Wolfram}
\includepdf[pages={1,2}]{pictures/mytask.pdf}


\subsection{Решение задачи о математическом осцилляторе в Wolfram Mathematica 9.}
\label{Wolfram}
\includepdf[pages={1}]{pictures/oscillator.pdf}


\end{document}