\section{Краевая задача}
Ко всему вышесказанному добавим, что 
\[
\int\limits_{0}^{1} x dt = 1.
\]
Введём обозначение
\[
\phi(t) = \int\limits_{0}^{t} x(\tau) d\tau, \quad \lambda_1 = a;
\]
тогда 
\begin{equation}\label{dopeqq}
\begin{cases}
\phi(0) = 0,\\ 
\dot\phi = x;\\
\phi(1) = 1;
\end{cases}
\end{equation}

Таким образом, на основе принципа максимума Понтрягина задача оптимального управления~(\ref{formtask}) сводится к~краевой задаче~(\ref{finaltask}). 
\begin{equation}\label{finaltask}\left\{
\begin{aligned}
	\dot\phi &= x,&\phi(0)&=0,\phi(1)=1;\\
	\dot{x} &= y,&x(0)&=0; \\
	\dot{y} &= p_y (1 + \alpha t^2x^2),&y(0)&=1,y(1)=0;\\ %т.к. \dot{y} = u
	\dot{p}_y &= - p_x;\\
	\dot{p}_x &= \lambda_1 = a,&p_x(1)&=0; \\
	\dot{a} &= 0.
\end{aligned}\right.
\end{equation}
Параметр $\alpha$ принимает следующие значения
\[ \alpha = \{0.0;\quad 0.1; \quad 1.0; \quad 10.0\}.\]

\section{Аналитическое решение краевой задачи}
Данная система уравнений не является аналитически интегрируемой. Тем не менее, $a = a=\const$, $p_x = a(t-1)$, $p_y = -\frac a2(t-1)^2 + b$, где $b=\const$. Далее неинтегрируемое уравнение второго порядка
\[
  \ddot x = \left( -\frac a2(t-1)^2 + b \right)(1+\alpha t^2 x^2),\quad
  \alpha,a,b= \const.
\]

При $\alpha=0$ имеем аналитическое решение. Имеем
\[
  y = -\frac a6(t-1)^3 + bt + c;
\]
Причем $y(1)=0$, отсюда $c = -b$. Далее $y(0)=1$ даёт $a - 6b = 6$.
\[
  x(t) = -\frac a{24}(t-1)^4 + \frac b2 t^2 - bt + \frac a{24};\quad
  \phi(t) = -\frac a{120}(t-1)^5 + \frac b6 t^3 - \frac b2 t^2 + \frac a{24}t + \frac a{120}.
\]
Условие $\phi(1)=1$ даёт 
\[
  -\frac b3 + \frac a{24} + \frac a{120} = 1\iff -\frac b3 + \frac a{20} = 1.
\]
Таким образом
\[
  -\frac b3 + \frac{3+3b}{10} = 1\iff  -b + 9 = 30 \iff b = -21, a = -120.
\]
\[\left\{
\begin{aligned}
  \phi(t) &=  (t-1)^5  - \frac72 t^3 + \frac{21}2 t^2 - 5 t -1,	& \phi(0)=0, \phi(1)=1;\\
  x(t)    &=  \frac{(t-1)^4}2 - \frac32 t^2 + 3 t,		& x(0)=0;\\
  y(t) &= 2(t-1)^3 - 3 t + 3,					& y(0)=1, y(1)=0;\\
  p_y(t) &= 6(t-1)^2 - 3;\\
  p_x(t) &= -12(t-1),						& p_x(1)=0;\\
  a &=  -12;
\end{aligned}\right.
\]
