\section{Краевая задача}
Ко всему вышесказанному добавим, что 
\[
\int\limits_{0}^{1} x dt = 1.
\]
Введём обозначение
\[
\phi(t) = \int\limits_{0}^{t} x(\tau) d\tau, \quad \lambda_1 = a;
\]
тогда 
\begin{equation}\label{dopeqq}
\begin{cases}
\phi(0) = 0,\\ 
\dot\phi = x;\\
\phi(1) = 1;
\end{cases}
\end{equation}

Таким образом, на основе принципа максимума Понтрягина задача оптимального управления~(\ref{formtask}) сводится к~краевой задаче~(\ref{finaltask}). 
\begin{equation}\label{finaltask}\left\{
\begin{aligned}
	\dot{x} &= y,&x(0)&=0; \\
	\dot{y} &= p_y (1 + \alpha t^2x^2),&y(0)&=1,y(1)=0;\\ %т.к. \dot{y} = u
	\dot{p}_x &= \lambda_1 = a,&p_x(1)&=0; \\
	\dot{p}_y &= - p_x;\\
	\dot\phi &= x,&\phi(0)&=0,\phi(1)=1;\\
	\dot{a} &= 0.
\end{aligned}\right.
\end{equation}
Параметр $\alpha$ принимает следующие значения
\[ \alpha = \{0.0;\quad 0.1; \quad 1.0; \quad 10.0\}.\]

\section{Аналитическое решение краевой задачи}
Полученная краевая задача решается аналитически.
\begin{enumerate}
\item из уравнения $\dot{p}_x = 1$ следует, что 
\[
p_x = t + C_1,
\] 
где $C_1 = \mathrm{const}$.\\
Так же из краевых условий видим, что $p_x(1) = 0$, тогда $0 = 1 + C_1$, значит $C_1 = -1$.
\item из уравнения 
\[
\dot{p}_y = - p_x = - t + 1
\]
 получим, что 
 \[
 p_y = -\frac{1}{2} t^2 + t + C_2,
 \qquad 
\text{где } C_2 = \mathrm{const}.
\]
\item  из уравнения 
\[
\dot{y} = p_y (1 + \alpha t^4) = (-\frac{1}{2} t^2 + t + C_2)(1 + \alpha t^4)
\]
 не сложно получить, 
\[
y = \frac{1}{2} \left(-\frac{2}{5} \alpha C_2 t^5-\frac{\alpha t^7}{7}-\frac{\alpha t^6}{3}-2 C_2 t-\frac{t^3}{3}-t^2\right)+C_3
\]
Из краевых условий $y(0) = 1$, $y(1) = 0$ получим
\[
C_3 = 1, \qquad C_2 = -\frac{5 (5 \alpha -7)}{21 (\alpha + 5)}.
\]
А значит 
\[
y = \frac{1}{42} \left(t \left(t \left(-3 \alpha t^5-7 \alpha t^4+\frac{2 \alpha(5 \alpha-7) t^3}{\alpha + 5}-7 t-21\right)-\frac{320}{\alpha + 5}+50\right)+42\right)
\]
\item из уравнения 
\[
\dot{x} = y = \frac{1}{42} \left(t \left(t \left(-3 \alpha t^5-7 \alpha t^4+\frac{2 \alpha(5 \alpha-7) t^3}{\alpha + 5}-7 t-21\right)-\frac{320}{\alpha + 5}+50\right)+42\right)
\]
следует
\[
x = \frac{t}{1008} \left(-9 \alpha t^7-24 \alpha t^6+\frac{8 \alpha (5 \alpha-7) t^5}{\alpha+5}+\frac{120 (5 \alpha-7) t}{\alpha+5}-42 t^3-168 t^2+1008\right) + C_4,
\]
Из краевых условий $x(0) = 0$, тогда
\[
C_4 = 0
\]
\end{enumerate}

Из вышесказанного следует, что решением нашей системы будет следующим
\[
	\begin{cases}
	x = \frac{t}{1008} \left(-9 \alpha t^7-24 \alpha t^6+\frac{8 \alpha (5 \alpha-7) t^5}{\alpha+5}+\frac{120 (5 \alpha-7) t}{\alpha+5}-42 t^3-168 t^2+1008\right); \\
	y = \frac{1}{42} \left(t \left(t \left(-3 \alpha t^5-7 \alpha t^4+\frac{2 \alpha(5 a-7) t^3}{\alpha + 5}-7 t-21\right)-\frac{320}{\alpha + 5}+50\right)+42\right);\\
	p_x = t - 1;\\
	p_y =  -\frac{1}{2} t^2 + t -\frac{5 (5 \alpha -7)}{21 (\alpha + 5)}
	\end{cases}
\]
